\documentclass[runningheads]{AIIT}
\usepackage[utf8]{inputenc}
\usepackage[russian]{babel}

\usepackage[usenames,dvipsnames,svgnames,table]{xcolor}
\usepackage{graphicx}
\definecolor{rclr}{rgb}{0.5,0.1,0.1}
\definecolor{eclr}{rgb}{0,0.5,0.5}
\colorlet{acolor}{blue}
\colorlet{rcolor}{red}
\definecolor{ncolor}{rgb}{0.5,0.5,0.1}
\newcommand{\aaa}[2][acolor]{\noindent\textcolor{eclr}%
{+\ [}\textcolor{#1}{#2}\textcolor{eclr}{]}}
\newcommand{\rrr}[2][rcolor]{\noindent%
\textcolor{eclr}{-\ [}\textcolor{#1}{#2}\textcolor{eclr}{]}}
\newcommand{\nnn}[2][rcolor]{\noindent%
\textcolor{eclr}{}\textcolor{#1}{#2}\textcolor{eclr}{}}

\usepackage{hyperref}
\hypersetup{
    bookmarks=true,         % show bookmarks bar?
    unicode=true,           % non-Latin characters in Acrobat’s bookmarks
    pdftoolbar=true,        % show Acrobat’s toolbar?
    pdfmenubar=true,        % show Acrobat’s menu?
    pdffitwindow=false,     % window fit to page when opened
    pdfstartview={FitH},    % fits the width of the page to the window
    pdftitle={Expert System for Structural Analysis\\ of Electrocardiogramms},    % title
    pdfauthor={Kristina Krasovitskaya, Evgeny Cherkashin},     % author
    pdfsubject={Electro cardio signal processing and recognition},   % subject of the document
    pdfcreator={EMACS-24.5:AuCTeX-0.89},   % creator of the document
    pdfproducer={PdfLaTeX}, % producer of the document
    pdfkeywords={Искусственный интеллект} {Логическое
      программирование} {Планирование действий} {Удовлетворение
      ограничений} {Компьютерная алгебра} {Принцип максимума} {Оптимальное управление}, % list of keywords
    pdfnewwindow=true,      % links in new window
    colorlinks=true,       % false: boxed links; true: colored links
    linkcolor=black,          % color of internal links (black)
    citecolor=black,        % color of links to bibliography
    filecolor=black,      % color of file links
    urlcolor=black           % color of external links
}


%% Necessary definitions for the running heads
\def\journalissue{International Conference on Applied Internet and Information Technologies, 2016}
\def\paperidnum{DOI: N/A}
\def\paperUDC{UDC: 551.4.01/02}
\setcounter{page}{1}

\title{Landscape Altitudinal Zonality Analysis Modules for Quantum GIS}
%  	 climatic zonality

%% Use this if the title is too long for the running heads
\titlerunning{Landscape zonality analysis modules}

\author{Evgeny Cherkashin\inst{1,2} \and Sergey Kuzmin\inst{3}, Irina Nevzorova\inst{4}, \nnn{S.} Shamanova\inst{3}}

%% Use this the list of authors is too long for the running heads
%\authorrunning{First Author et al.}

\institute{Institute of System Dynamics and Control Theory at Siberian Branch of Russian Academy of Sciences,\\
Lermontov str. 134, Irkutsk, 664033, Russian Federation
  \and
National Research Irkutsk State Technical University,\\
Lermontov str. 83, Irkutsk, 664078, Russian Federation\\
  \email{eugeneai@icc.ru}
\and
V.B.Sochava Institute of Geography at Siberian Branch of Russian Academy of Sciences,\\
Ulan--batorskaya str. 1, Irkutsk, 664033, Russian Federation\\
  \email{?????}
  \and
Limnological Institute at Siberian Branch of Russian Academy of Sciences,\\
Ulan--batorskaya str. 3, Irkutsk, 664033, Russian Federation\\
  \email{?????}}
\begin{document}
%\selectlanguage{}
\maketitle

\begin{abstract}
\nnn{The paper deals with\ldots}
\vspace{6pt}\textbf{Keywords:} Lanscape structure analysis, geoinformational system, GRID-data
\end{abstract}

\section{Introduction}
\label{sec:introduction}

Geoinformational cartography and modeling are essential concepts of digital relief model (DRM) processing that is carried on by means of remote sensing methods.  The processing aimed at landscape and ecological condition assessment, and DRM are the main source of data prepared with GIS software.  Geoinformational cartography allows us to carry out multilevel landscape analysis, modeling geostatic and geodynamic features of field testing sites, manipulate the DRM data itself, as well as its derivative interpretation maps.

In \emph{geoinformational modeling} (GM) the relief is represented as a 3D model with interpolated topographical values.  The quality of GM processing depends on the correctness degree of hypsographic surface smoothness and variability on the different morphostructural levels.  The generic way of DRM preparation is an interpolation surface construction on the base of digitized isolines of topographic maps.  The resulting DRM is a matrix of 3D--points.  The following algorithmic analysis of the DRM will obtain new data that is inaccessible in field studies.

The structural analysis of DRM is based on relief surface method (RSM).  It describes the Earth topology with altitude isolines, horizontal, vertical and average Gaussian curvatures, as well as curvature extreme relief curve lines (separating ridge, talweg, foot of hill, shoulder).  Thus, the isolines are the source information for analysis of the relief invariant structures, floors and their altitude landscape zonality.  RSM is based on isoline second derivative extreme points calculation.  New cartographic isoline \emph{morphoisograph} introduced that represents plane and two--dimensional profiling.  The isoline connects inflection points of neighbor isohypses and structures the Earth surface into areas of divergence and convergence of substance flows.

RSM recognizes structural levels (floors) of a relief and implements a morphostructural approach.  The structural levels, in turn, are gathered into a system structure of interconnected floors, where connection expresses the intensities of substances flows between the floors.  New GIS layers are synthesized as a result of the morphostructural analysis applications.  The key landscape features of the DRM are verified at the testing sites and by geoimaging (aero and space).  The joint application of RSM as geographic comparative method with other methods results in general description of the relief as an invariant geographic system. \nnn{comment the invariant, what is it?}

\aaa{About problem of raster representation of the source DRM}.

\section{RSM for GRID data}
\label{sec:rsm-grid-data}

For the case when the relief surface is represented as GRID, i.e., 2D array of relief heights, and the isolines are not accessible, we developed new restively simple technique of RSM based on local gradients assessment.  The main idea is implied from the notion of RSM: the surface decided into two areas of convergence and divergence.

Consider Fig.~\ref{fig:1}.  The arrows in the cells denote vector projections of the antigradients in the cells.  In the convergence area (Fig.~\ref{fig:1}\emph{a}) the vectors are converge, and in the divergence area (Fig.~\ref{fig:1}\emph{b}) diverge.  The RSM is determined by processing relief GRID matrix with a $3 \times 3$ scanning window.  To recognize the two areas one are to juxtapose vector projections to a template of unit vectors showed in Fig.~\ref{fig:1}\emph{c} by their scalar multiplication and summation of the four values.  If the sum is positive, then the central point belongs to a divergence area; if the sum is negative, then the area belongs to convergence area.

\begin{figure}[htb]
  \centering

  \caption{Technique of relief surface method for GRID data}
  \label{fig:1}
\end{figure}



Объекты и процедура исследований

Объектом исследований выбрана территория Приольхонья, которая расположена в Западном Прибайкалье (Рис.1). Она ограничивается береговой линией озера Байкал и водоразделом Приморского хребта между поселками Бугульдейка и Зама. Приольхонье почти полностью входит в состав Косостепско-Приольхонского округа, Прибайкальской провинции, Байкальско-Становой страны. Это определяет особенности климата и геолого-геоморфологического строения территории. Подробное геоморфологическое описание района исследований дано в многочисленных специальных работах [12-23]. Собственно моделируемый полигон расположен в районе залива Мухор между верховьями реки Кучелга и долиной реки Харга, и включает приводораздельные участки Приморского хребта, участок его макросклона, фрагменты Кучелго-Таловской депрессии и Приольхонского плато.


Процедура моделирования начинается с составления цифровой модели местности. Под цифровой моделью местности в нашей работе понимается определенная форма представления исходных данных и способ их структурного описания, позволяющий «вычислять» (восстанавливать) объект (рельеф) путем интерполяции, аппроксимации или экстраполяции [3]. Моделирование рельефа осуществлялось путем интерполяции оцифрованных изолиний с топографических карт, при этом, следует сказать, что варианты моделирования могут быть разгруппированы, исходя из принципа моделирования. Модели могут быть представленные в двух видах: TIN и GRID.
TIN – система неравносторонних треугольников, соответствующая триангуляции Делоне. Она используется в качестве модели данных при конструировании цифровой модели рельефа, представляя его набором высотных отметок в узлах сети, и заменяя его, тем самым, многогранной поверхностью. Кроме этого, модели TIN могут использоваться при генерации дополнительных данных при их нехватке для интерполяции. Преимущество триангуляционной модели заключается в отсутствии преобразований над исходными данными, но это не позволяет использовать ее для детального анализа.
GRID – представление цифровой модели в виде регулярной сетки квадратов, когда в ее узлах заданы значения показателя. Модели, полученные при интерполяции таким способом, представляют собой непрерывную матрицу данных, которая может быть подвергнута более тщательному анализу, поэтому данный способ моделирования и представления данных и был нами использован для создания цифровой модели рельефа.

Составление цифровых карт происходило в среде ГИС ArcInfo и ArcView и состояло из следующих основных этапов.

Сканирование на высокоточных планшетных сканерах.

Взаимоувязка  фрагментов. При использовании малоформатных сканеров карту сканируют фрагментами. Соседние фрагменты должны иметь перекрытия, необходимые для последующей «склейки». Из опыта работ можно сказать, что перекрытия соседних фрагментов карты должны составлять не менее 10-12 см. Эта процедура выполняется в программе EasyTrace.

Привязка растров. Для этого используются пересечения линий прямоугольной координатной сетки. Оптимальное количество контрольных точек для каждого фрагмента карты составляет 4-9 штук.
Векторизация. Осуществляется путем ручной или полуавтоматической (в зависимости от сложности карты) трассировки по растровой подложке с выделением отдельных тематических слоев. Каждый слой несет информацию об одном из аспектов исследуемой территории. В результате формируется точечная, либо линейная карта объектов. Векторизацию проводят в программах-векторизаторах: EasyTrace или R2V.

Экспорт векторных слоев в формат ГИС ArcView.

«Сшивка» карт. Две изолинии на разных картах проверяются на наличие одинаковых идентификаторов, а затем с помощью встроенных функций программы соединяются в одну изолинию.
ArcView – ГИС, обладающая необходимым набором средств для ввода, хранения и обработки пространственных данных. Функциональный набор операций ArcView расширяется при помощи множества дополнительных модулей, как внешних, так и реализованных в виде скриптов языка программирования Avenue. С помощью дополнительного модуля Spatial Analyst производится обработка данных, представленных в виде грид-поверхностей, и осуществляются запросы к этим данным. Кроме того, существует значительно обогащающие возможности ArcView модуль X-Tools. При помощи X-Tools выполняются типичные редакторские операции с темами.

Для анализа пластики рельефа, представленного в виде грида, использовалась программа OpenEV. Встроенный в программу инструментарий позволяет осуществлять программируемые в виде скриптов матричные преобразования над гридами. Программирование производилось в среде Python, состоящей из языка программирования и библиотеки функциональных модулей. Добавление модулей преобразования грид-данных – трудоемкая задача. OpenEV позволяет экспортировать получаемые грид-данные в большинство грид-форматов, в том числе в форматы, поддерживаемые ГИС ArcView.
В программе R2V оцифрован исследуемый участок местности на картах масштаба  1 : 25 000, отсканированных и сохраненных в формате *.tif. Выделены: изолинии высот, реки, береговая линия, высотные отметки и др.

Морфометрический анализ рельефа, представленного в виде грид-данных, отличается от анализа, использующего в качестве исходных данных горизонтали морфоизогипс. В основу анализа положено утверждение о том, что все точки грида разделяются на два класса: конвергенции и дивергенции. Считается, что точка принадлежит области (классу) конвергенции, если расположенные в окрестности этой точки линии стока (вектора) сходятся. Точка принадлежит области дивергенции, если расположенные в окрестности этой точки линии стока (вектора) расходятся [5].
При помощи библиотеки GDAL производится расчет поля векторов стока как приближение отношений разности уровней точек окрестности к расстоянию между этими точками (принята система координат, где ось y направлена вниз):








Далее анализируется схождение/расхождение векторов стока в окрестности точки, которое производится следующим образом. Векторы стока расходятся, если разность сумм проекций векторов стока снизу и справа, сверху и слева от центральной точки на ось x и y положительна. Если указанная сумма отрицательна, то векторы стока сходятся.
Процедура моделирования рельефа на основе метода пластики рельефа сводится к следующим шагам (подпрограммам).

Вычисление поля векторов стока. В результате формируются два массива: sx – x-составляющие поля векторов стока, sy – y-составляющие поля векторов стока. Составляющие поля векторов стока являются грид-данными.

Вычисление схождения/расхождения векторов стока. Вычисляется разность сумм проекций x-составляющих и y-составляющих векторов стока.
Анализ поля схождения/расхождения векторов стока. На поле схождения/расхождения, выделяются области, соответствующие знаку величины схождения/расхождения. Анализ осуществляется сравнением величины с 0.

Для обработки исходных GRID-данных, полученных при помощи интерполяции изолиний используется обобщение методики, описанной выше. Такие данные содержат искажения, оказывающие существенное влияние на качество конечного результата. Поле схождения/расхождения векторов стока вычисляется не по четырем векторам, а по дискретному контуру, являющемуся аппроксимацией окружности заданного радиуса с центром в искомой точке.

Окружность строится при помощи алгоритма Браземхэма. Вектор стока в точке окружности скалярно умножается на единичный вектор, направленный из этой точки в центр окружности, результаты умножения для всех точек окружности суммируются и усредняются. Дальнейшая процедура анализа пластики рельефа проводится аналогично.

Параметром к этой обобщённой методики является радиус окружности (в точках). Чем больше радиус, тем сильнее будет происходить усреднение значений, т.е. уменьшение влияния искажений GRID на конечный результат, однако разрешение искомого результата также уменьшается в количество раз равному радиусу окружности.

Геоморфологический анализ карт пластики рельефа проведем с помощью двух описанных выше моделей, которые для простоты изложения назовем условно «рельефX» и «рельефY» (Рис.2,3). Они отражают соответственно горизонтальную (плановую) и вертикальную (профильную) кривизну земной поверхности. Раскраска грид-данных отображает интенсивность стока по соответствующей координате в конкретной точке. Анализ будет проведен только для изображенной на рисунках части исследуемой территории Приольхонья. Экстраполяция данных – задача последующих исследований.

Анализ моделей по ярусам рельефа (структурный)

Водораздельный ярус представлен гривами и гребнями отрогов Приморского хребта, осложненными останцовыми формами. Их осевые линии дешифрируются с помощью модели «рельефX». Модель «рельефY» позволяет проследить главные направления сноса осадочного материала с гребней и грив, а также оценить площади и конфигурацию поверхностей сноса.

Склоновый ярус рельефа представлен приводораздельным пологим макросклоном Приморского хребта, крутым склоном-эскарпом вдоль Приморского разлома, пологим предгорным делювиальным шлейфом, а также отдельными небольшими склонами на участке Приольхонского плато.

Приводораздельный макросклон проявлен на обеих моделях рельефа достаточно отчетливо, причем заметно, что при приближении к склону-эскарпу, макросклон становится все более террасированным, с резкими (пусть и небольшими) перепадами высот. Отследить эти перепады с помощью топографических карт и даже при визуальном обследовании практически невозможно. Более того, модель «рельефY» позволяет на пологой верхней поверхности выделить зоны (полосы) преимущественной аккумуляции (заболоченные пространства) и сноса рыхлого материала, а на нижней террасированной поверхности даже подсчитать количество невысоких уступов (перепадов) и определить их морфометрические характеристики, что также возможно только с использованием данных моделей пластики рельефа. Модель «рельефX» позволяет проследить переходы уступов на водораздельных гривах и гребнях в перепады на макросклоне.
Склон-эскарп Приморского разлома также отчетливо проявлен на обеих моделях. Модель «рельефY» позволяет определить ориентацию и приблизительную ширину склона-эскарпа. Модель «рельефX» в этом смысле более информативна. Кроме указанных морфометрических характеристик, она позволяет определить количество отдельных уступов на склоне-эскарпе, их ширину, ориентацию и геометрические взаимоотношения (которые являются отражением позднекайнозойских тектонических движений). Эта же модель позволяет выделить делли (зарождающиеся распадки) на склоне-эскарпе, что в других случаях возможно только при визуальном обследовании.
Предгорный делювиальный склон (шлейф) Приморского разлома проявлен на моделях фрагментарно, поскольку прерывается по простиранию боковыми долинами-притоками реки Кучелга. Обе модели подчеркивают две разновозрастных генерации поверхностей делювиального склона-шлейфа. Более древняя генерация значительно положе и сохранилась только в виде небольших участков, слабо- или совсем нетеррасированных (может быть сопоставлена с эпохой среднеплейстоценового относительного тектонического покоя). Более молодая генерация круче и заметно террасирована (что особенно хорошо проявлено в модели «рельефY»). На ней отмечаются закладывающиеся совсем юные эрозионные врезы, которые можно сопоставить с эпохой тектонической активизации второй половины позднего плейстоцена.

Склоны Приольхонского плато также хорошо проявлены на обеих моделях и их анализ возможен для конкретных участков. Модель «рельефY» подходит для анализа экспозиционных различий склонов. Модель «рельефX» наиболее подходит собственно для морфометрического анализа: протяженность, ширина, ориентация и другие характеристики склонов (например, схождения, узловые сочленения, трансекции и др.).

Долинный ярус представлен террасами, высокой и низкой поймой реки Кучелга с левыми безымянными притоками и реки Харги.

Граница долинного яруса наиболее четко прослеживается на модели «рельефX». Выделяются крупные долины постоянных водотоков, трассирующие всю территорию, верховья которых расположены на Приморском хребте, и мелкие долины временных водотоков на Приольхонском плато. Модель «рельефY» позволяет выделить в долинном ярусе фрагменты террас, высокой и низкой (в основном заболоченной) поймы.

Аквальный ярус представлен подводными формами залива Мухор на модели «рельефY». Это дельтовые наносы в устьевой части реки Кучелга, волноприбойные косы на мысах, бар в волновой «тени» острова Тойнак.

Анализ моделей по происхождению рельефа (генетический)

По генезису преобладает эрозионно-денудационный, тектонический, гравитационный, карстово-суффозионный и эоловый рельеф. Широко представлены формы рельефа смешанного генезиса.
Эрозионно-денудационный рельеф дешифрируется на обеих моделях в виде долин постоянных и временных водотоков. Более информативна в этом смысле модель «рельефX». Эти геоморфологические элементы достаточно подробно рассмотрены в предыдущем разделе. Здесь отметим только, что среди денудационных форм выделяются фрагменты поверхностей выравнивания, древних структурно-террасовых уровней и собственно речных долин (дочетвертичных или среднеплейстоценовых), отрезанных от современного базиса эрозии и эрозионной сети, и разрушающихся ныне в основном за счет гравитационных процессов (подробнее будут рассмотрены ниже).

Тектонический рельеф представлен широким многообразием форм за счет того, что исследуемый участок расположен в зоне активного Приморского разлома. Морфоструктурный анализ проведен в основном по модели «рельефX». Прежде всего, отмечается собственно осевая зона Приморского разлома. Она сложно построенная, состоит из серии более мелких тектонических уступов, западин и шерлопов. Характер рельефа свидетельствует о том, что Приморский разлом является сбросом. Вместе с тем, отмечается наличие сдвиговой компоненты в кинематике тектонических движений по разлому.
Ранее нами отмечалась только лево-сдвиговая компонента [19]. В работах С.И. Шермана [24,25] высказывалась мысль о право-сдвиговых движениях по разлому. Анализ модели «рельефX» приводит нас к мысли о переменном знаке сдвиговых дислокаций по разлому, по крайней мере в плиоцен-четвертичное время. На модели «рельефХ» наиболее крупные долины (плиоцен?) имеют левый сдвиг, менее крупные долины (ранний-средний плейстоцен?) имеют правый сдвиг, и, наконец, самые мелкие и молодые элементы рельефа (поздний плейстоцен?) вновь приобретают левый сдвиг.
Хорошо проявлены и другие древние геолого-структурные швы, например, зона Чернорудско-Баракчинского разлома. В восточной части исследуемого участка она подчеркнута небольшими линейно вытянутыми эскарпами. Особенно примечателен Мухорский эскарп, являющийся почти точной копией эскарпа Приморского разлома в миниатюре. Анализ структурного рисунка зоны Мухорского и параллельных ему эскарпов позволяет также говорить о наличии лево-сдвиговой компоненты в самых молодых рифтогенных разломах Приольхонья. На модели «рельефХ» впервые однозначно зафиксированы правосторонние сдвиги по разломам, поперечным основным рифтовым структурам (например, на севере участка вдоль молодой гривы).
Модель «рельефХ» позволяет оконтурить тектонические блоки, разделенные активными разломами. Иногда можно определить кинематику и характер тектонических движений блока. Например, на западе участка правым сдвиго-сбросом вычленен узкий полосообразный блок, испытывающий активное опускание, совмещенное с кручением по часовой стрелке.
Более детальный анализ на отдельных фрагментах позволяет выделить и другие, не менее интересные тектонически обусловленные формы рельефа.

Гравитационный рельеф во многом повторяет тектонический, хотя встречаются и своеобразные геоморфологические элементы. Прежде всего, это шлейфы рыхлого материала у подножья эскарпов. Гравитационное разрушение эскарпов протекает настолько активно, что мощность шлейфов даже под невысокими эскарпами достигает больших величин. Например, у подножья Мухорского эскарпа высотой всего 20-30 м мощность шлейфа превышает 1.8 м (заложенный шурф на этой глубине не встретил кристаллических пород, что позволяет предполагать и большую величину мощности рыхлых отложений). Оползневые гравитационные структуры дешифрируются на обеих моделях на предгорном делювиальном шлейфе Приморского хребта. Обвально-осыпные формы отмечаются на модели «рельефХ» в продольных профилях распадков, поперечных р.Кучелга. На модели «рельефY» фиксируется конус выноса крупно-глыбового материала в долине р.Харга.

Карстово-суффозионный рельеф отмечается на участке Приольхонского плато на модели «рельефХ». В основном это провальные котловины и небольшие суффозионные воронки. В первом случае следует отметить также карстовые пещеры. В последнем случае невозможно обойти вниманием карстовые провалы, совмещенные со складчатыми структурами.

Эоловый рельеф проявлен дефляционными формами в виде котловин выдувания и останцов обтачивания на участке Приольхонского плато. Ярко выраженный останец обтачивания расположен в центральной части, а две котловины выдувания – в восточной части исследуемой территории.

Рельеф смешанного генезиса выделяется на обеих моделях. Формы его разнообразные, поэтому приведем лишь несколько примеров. Упомянутые выше котловины выдувания сформированы над карстовыми внутригорными полостями, в которые происходили гравитационные провалы, а затем эти элементы оформлялись дефляцией. Рядом с котловинами, несколько юго-западнее расположена карстово-тектоническая замкнутая депрессия в виде грабена. Здесь внутригорная карстово-суффозионная полость способствовала тектоническому обрушению прямоугольного блока. Причем это обрушение произошло сравнительно недавно, поскольку на дне депрессии еще не успел сформироваться слой рыхлого материала. Флювиально-гравитационно-тектонические оползни дешифрируются на предгорном шлейфе Приморского хребта, где их подновлению периодически способствуют землетрясения. Прибрежные косы и бары залива Мухор формируются из песчаного материала, поставляемого, главным образом, эоловыми процессами.

Анализ реликтовых форм рельефа

Модель «рельефХ» и отчасти модель «рельефY» позволяют анализировать реликтовые формы рельефа, такие как поверхности выравнивания, древние террасовые уровни, участки брошенных древних речных долин.

Фрагмент раннеплиоценовой (?) поверхности выравнивания прослеживается на обеих моделях в приводораздельной части Приморского хребта. На севере имеется еще один менее выразительный фрагмент этой поверхности. Хотя, по всей видимости, вся приводораздельная часть Приморского хребта (с гривами, пологими склонами и выровненными поверхностями) может рассматриваться как реликт раннеплиоценового рельефа. Этого же возраста поверхности выравнивания прослеживаются и на участке Приольхонского плато, где они занимают небольшие площади и расположены мозаично.
Реликтовые террасовые уровни (два?) прослеживаются в восточной части исследуемой территории, где они привязаны, вообще говоря, к брошенной древней речной долине. Террасы не только структурные, но и аккумулятивные, о чем свидетельствует мощная толща (более 1 м) рыхлых отложений песчаного и песчано-гравелистого материала на их поверхности. Возраст террас и долины можно сопоставить с эпохой тектонического покоя среднего плейстоцена. Сток по этой древней брошенной долине осуществлялся с северо-востока на юго-запад. В современной речной сети такое направление стока для территории Приольхонья не встречается. Это позволяет думать, что фаза тектонической активизации позднего плейстоцена привела к существенным геоморфологическим преобразованиям на исследуемом участке.

Северо-западнее, на предгорном шлейфе Приморского хребта, в седловине между двумя распадками также отмечаются реликты среднеплейстоценового (?) рельефа. С юго-запада это фрагмент поверхности выравнивания. С северо-востока, на склоне седловины, прилегающем к долине р. Харга, отмечаются реликтовые террасовые уровни (вероятно только структурные, поскольку обнаружить на них аллювий не удалось). Можно проследить даже излучину (меандр) древней реки. Террасовых уровней здесь, по крайней мере, три, и там, где заканчивается самый нижний из них, расположен резкий крутой (обрывистый) склон к современной долине р. Харга.

Реликты древних долин имеются и в южной части исследуемой территории, где они также подтверждают существование еще в раннем плейстоцене речного стока с северо-востока на юго-запад. Поскольку они небольшие, то нет основания распространять данный вывод на все Приольхонье, пока не будут получены дополнительные материалы на других участках.
Заключение

Таким образом, геоинформационное моделирование на основе метода пластики рельефа показало его высокую информативность для проведения геоморфологических исследований. Следует отметить главную особенность моделей «рельефX» и «рельефY», которая заключается в том, что на первой модели более четко проявлены индикационные признаки эндогенного (тектонического) рельефа, а на второй – экзогенного рельефа (хотя элементы всех типов рельефа имеются и на обеих моделях). Модели могут использоваться как совместно, так и по отдельности (в зависимости от конкретной задачи), но в любом случае они должны рассматриваться как взаимодополняющие друг друга при дистанционных геоморфологических построениях.

Список литературы

1. Коновалова Н.А., Капралов Е.Г. Введение в ГИС. Петрозаводск: Изд-во ПГУ, 1995. 148 с.
2. Жуков В.Т., Новаковский Б.А., Чумаченко А.М. Компьютерное геоэкологическое картографирование. М.: Научный мир, 1999. 128 с.
3. Новаковский Б.А., Прасолов С.В., Прасолова А.И. Цифровые модели рельефа реальных и абстрактных геополей. М.: Научный мир, 2003. 64 с.
4. Степанов И.Н., Абдуназаров У.К., Брынских М.Н. и др. Временная методика по составлению карт пластики рельефа крупного и среднего масштаба. Методические рекомендации. Пущино: ОНТИ НЦБИ АН СССР, 1983. 112 с.
5. Метод пластики рельефа в тематическом картографировании. Пущино: ОНТИ НЦБИ АН СССР, 1987. 160 с.
6. Геометрия структур земной поверхности. Пущино, 1991. 202 с.
7. Ласточкин А.Н. Рельеф земной поверхности. Ленинград: Недра, 1991. 240 с.
8. Симонов Ю.Г. Объяснительная морфометрия рельефа. М.: ГЕОС, 1999. 263 с.
9. Пириев Р.Х. Методы морфометрического анализа рельефа. Баку: Элм, 1986. 117с.
10. Поздняков А.В., Черванев И.Г. Самоорганизация в развитии форм рельефа. М.: Наука, 1990. 202 с.
11. Якименко Э.Л. Морфометрия рельефа и геология. Новосибирск: Наука, 1990. 201 с.
12. Логачев Н.А, Ломоносова Т.К., Климанова В.М. Кайнозойские отложения Иркутского амфитеатра. М.: Наука, 1964. 195 с.
13. Флоренсов Н.А. Байкальская рифтовая зона и некоторые задачи ее изучения // Байкальский рифт. М.: Наука, 1968. С.40-56.
14. Нагорья Прибайкалья и Забайкалья. М: Наука, 1974. 360 c.
15. Плешанов С.П., Ромазина А.А. Основные этапы формирования рельефа Приольхонья // Геоморфология. 1975. № 4. С.85-89.
16. Тайсаев Т.Т. Эоловые процессы в Приольхонье и на о.Ольхон (Западное Прибайкалье) //  Доклады АН СССР. 1982. № 4. С.948-951.
17. Уфимцев Г.Ф. О неотектонике Приольхонья // Геология и геофизика. 1985. № 6. С.37-45.
18. Васянович А.В. Геоморфологические особенности территории Прибайкальского национального парка // География и природные ресурсы. 1990. № 4. С.67-77.
19. Кузьмин С.Б. Геоморфология зоны Приморского разлома (Западное Прибайкалье) // Геоморфология. 1995. № 4. С.53-61.
20. Кузьмин С.Б. Инженерно-экологическая оценка рельефа Приольхонья в рекреационно-туристических целях // Инженерная экология. 2002. № 6. С.41-53.
21. Макаров С.А. Геоморфологические процессы Приольхонья в голоцене // География и природные ресурсы. 1997. № 1. С.77-84.
22. Аржанникова А.В., Гофман Л.Е. Проявления неотектоники в зоне влияния Приморского разлома // Геология и геофизика. 2000. № 6. С.811-818.
23. Снытко В.А., Данько Л.В., Кузьмин С.Б. и др. Разнообразие геосистем контакта тайги и степи западного побережья Байкала // География и природные ресурсы. 2001. № 2. С.61-68.
24. Шерман С.И. Приморский разлом в Западном Прибайкалье // Информационный бюллетень ИЗК СО АН СССР (1967-1968 г.г.). Иркутск, 1970. С.14-15.
25. Шерман С.И. Физические закономерности развития разломов земной коры. Новосибирск: Наука, 1977. 101 с.






Подписи к рисункам

Рис. 1. Район исследований – Приольхонье (показан штриховкой).

Рис. 2. Цифровая модель пластики рельефа Приольхонья – рельеф X.

Рис. 3. Цифровая модель пластики рельефа Приольхонья – рельеф У.

%-----------------------------------

%\bibliographystyle{splncs03}
%\bibliography{example}
 \begin{thebibliography}{99}

 \end{thebibliography}

%%%%%%%%%%%%%%%%%%%%%%%%%%%%%%%%%%%%%%%%%
%% For the final version of the paper: %%
%%%%%%%%%%%%%%%%%%%%%%%%%%%%%%%%%%%%%%%%%

%% Author information
%\vspace{4ex}\noindent
%\textbf{Author One} is\dots
%
%\bigskip\noindent
%\textbf{Author Two} is\dots
%
%\bigskip\noindent
%\textbf{Author Three} is\dots

%% Reception and acceptance information
%\bigskip
%\paragraph{Received: Month DD, 20YY; Accepted: Month DD, 20YY.}

\end{document}

%%% Local Variables:
%%% mode: latex
%%% TeX-master: t
%%% End:
