\documentclass[runningheads]{AIIT}
\usepackage[utf8]{inputenc}
\usepackage[russian]{babel}

\usepackage[usenames,dvipsnames,svgnames,table]{xcolor}
\usepackage{graphicx}
\definecolor{rclr}{rgb}{0.5,0.1,0.1}
\definecolor{eclr}{rgb}{0,0.5,0.5}
\colorlet{acolor}{blue}
\colorlet{rcolor}{red}
\definecolor{ncolor}{rgb}{0.5,0.5,0.1}
\newcommand{\aaa}[2][acolor]{\noindent\textcolor{eclr}%
{+\ [}\textcolor{#1}{#2}\textcolor{eclr}{]}}
\newcommand{\rrr}[2][rcolor]{\noindent%
\textcolor{eclr}{-\ [}\textcolor{#1}{#2}\textcolor{eclr}{]}}
\newcommand{\nnn}[2][rcolor]{\noindent%
\textcolor{eclr}{}\textcolor{#1}{#2}\textcolor{eclr}{}}

\usepackage{hyperref}
\hypersetup{
    bookmarks=true,         % show bookmarks bar?
    unicode=true,           % non-Latin characters in Acrobat’s bookmarks
    pdftoolbar=true,        % show Acrobat’s toolbar?
    pdfmenubar=true,        % show Acrobat’s menu?
    pdffitwindow=false,     % window fit to page when opened
    pdfstartview={FitH},    % fits the width of the page to the window
    pdftitle={Expert System for Structural Analysis\\ of Electrocardiogramms},    % title
    pdfauthor={Kristina Krasovitskaya, Evgeny Cherkashin},     % author
    pdfsubject={Electro cardio signal processing and recognition},   % subject of the document
    pdfcreator={EMACS-24.5:AuCTeX-0.89},   % creator of the document
    pdfproducer={PdfLaTeX}, % producer of the document
    pdfkeywords={Искусственный интеллект} {Логическое
      программирование} {Планирование действий} {Удовлетворение
      ограничений} {Компьютерная алгебра} {Принцип максимума} {Оптимальное управление}, % list of keywords
    pdfnewwindow=true,      % links in new window
    colorlinks=true,       % false: boxed links; true: colored links
    linkcolor=black,          % color of internal links (black)
    citecolor=black,        % color of links to bibliography
    filecolor=black,      % color of file links
    urlcolor=black           % color of external links
}


%% Necessary definitions for the running heads
\def\journalissue{International Conference on Applied Internet and Information Technologies, 2016}
\def\paperidnum{DOI: N/A}
\def\paperUDC{UDC: 551.4.01/02}
\setcounter{page}{1}

\title{Landscape Altitudinal Zonality Analysis Modules for Quantum GIS}
%  	 climatic zonality

%% Use this if the title is too long for the running heads
\titlerunning{Landscape zonality analysis modules}

\author{Evgeny Cherkashin\inst{1,2} \and Sergey Kuzmin\inst{3}, Irina Nevzorova\inst{4}, \nnn{S.} Shamanova\inst{3}}

%% Use this the list of authors is too long for the running heads
%\authorrunning{First Author et al.}

\institute{Institute of System Dynamics and Control Theory at Siberian Branch of Russian Academy of Sciences,\\
Lermontov str. 134, Irkutsk, 664033, Russian Federation
  \and
National Research Irkutsk State Technical University,\\
Lermontov str. 83, Irkutsk, 664078, Russian Federation\\
  \email{eugeneai@icc.ru}
\and
V.B.Sochava Institute of Geography at Siberian Branch of Russian Academy of Sciences,\\
Ulan--batorskaya str. 1, Irkutsk, 664033, Russian Federation\\
  \email{?????}
  \and
Limnological Institute at Siberian Branch of Russian Academy of Sciences,\\
Ulan--batorskaya str. 3, Irkutsk, 664033, Russian Federation\\
  \email{?????}}
\begin{document}
%\selectlanguage{}
\maketitle

\begin{abstract}
The paper deals with process of a cardiological expert system development. A definition of a electrocardiogram is presented. Problems of ECG characteristics determination such as ECG data digitizing is considered. A problem of QRS complex recognition and P and T waves parameters measurement is discussed. A general outline of analysis technique for ECG using wavelet transformation are proposed.

\vspace{6pt}\textbf{Keywords:} electrocardiogram,  time series recognition, image processing, continuous wavelet transformation.
\end{abstract}

%-----------------------------------



Е.А. Черкашин1,

1Институт динамики систем и теории управления СО РАН, Иркутск, Россия

…
Геоинформационное картографирование и моделирование рельефа является важным звеном в цепи дистанционных методов, направленных на оценку ландшафтной и экологической ситуации. Исходными данными для этого являются цифровые модели рельефа (ЦМР), которые строятся при помощи программных средств ГИС. Геоинформационное картографирование позволяет проводить разноуровенный ландшафтный анализ, моделировать геостатику и геодинамику на модельных полигонах, оперировать как самими ЦМР, так и построенными на их основе интерпретационными картами.
Геоинформационное моделирование позволяет создать трехмерную модель рельефа на основе интерполированных топозначений, корректно передать уровень гладкости и вариативности гипсометрической поверхности на разных морфоструктурных уровнях. Большое значение имеет также скорость обработки вычислительного алгоритма и наличие достаточного количества геокодированных точек измерений, чтобы с высокой точностью провести анализ геометрии и топологии рельефа. Проводится корректировка с традиционными методами, узловые участки проверяются на основе материалов дистанционного зондирования (ДЗЗ), данных СРТМ-формата и полевых GPS-точек.
Геоинформационное моделирование рельефа осуществлялось путем интерполяции оцифрованных изолиний с топографических карт, а варианты матрицы ЦМР разгруппированы, исходя из самого принципа моделирования: TIN, GRID или TGRID. Мы использовали метод создания ЦМР путем векторизации рельефа. Он в отличие от SRTM-формата позволяет получить корректные параметры морфометрии рельефа без заложения погрешностей на высоту 3-х ярусного растительного покрова. В этом случае не нужны поправки на неточное определение введенных плановых координат в горных областях, где погрешность составляет до 30 м, в долинах рек – до 20 м. Специфические единицы измерения SRTM также требуют адаптационных действий для трансляции обменных файлов данных. Используется MIF/MID MapInfo Data Interchange Format – открытый обменный формат MapInfo, а также шейп-файлы (shapefiles) ГИС-пакетов фирмы ESRI.
Использованные на этом этапе элементы модернизации создают завершенную ЦМР, которая позволяет оперировать данными путем объединения пространственной и позиционно-атрибутивной информации в 3D-модели рельефа с 1-2 точками проектирования и интерполяции GRID-поверхности рельефа на матрицу с размером ячеек 25 м. Векторные слои сливаются в структурно-измененную, перестроенную таблицу с добавленными полями значений, образуя сплошной массив точек. Для создания грида использовались как объектно-ориентированные, так и битовые файлы, геокодированные в универсальной поперечной проекции Меркатора (UTM). В результате геоинформационного анализа модернизированной ЦМР получены такие данные, которые ранее были недоступны даже при натурных исследованиях.
В основу последующего математического моделирования ЦМР положены принципы метода пластики рельефа – МПР. Он описывает топологию земной поверхности с помощью изолиний высоты, горизонтальной, вертикальной и средней гауссовой кривизны, а также экстремумов кривизны – структурных линий рельефа (водораздел, тальвег, подошва, бровка). Поэтому информация, заложенная в горизонталях топографических карт, становится приемлемой для морфометрического и динамического анализа инвариантной структуры рельефа, его ярусов и основанных на них высотных поясов ландшафта. МПР основан на геометрическом преобразовании изогипс на основе «метода вторых производных». Вводится новая картографическая изолиния – плановой и профильной кривизны – морфоизографа, которая соединяет точки перегибов изогипс с одинаковой кривизной (нулевой). Она структурирует земную поверхность путем разделения на области дивергенции и конвергенции потоков вещества, создающие потоковые структуры.
Области дивергенции и конвергенции вещества отображают ориентированные гравитационным полем Земли поверхности – относительные повышения и понижения в рельефе. При этом решается задача количественного картографического представления топографических факторов перераспределения вещества, его сноса, миграции и аккумуляции. Причем МПР даже на равнинных территориях позволяет выделить и изучить систему бассейнов стока, в пределах которых прослеживаются области формирования, транзита и аккумуляции твердого и водного стока. МПР позволяет также выявлять бассейны стока по следам древней гидрографической сети.
В основе МПР лежит морфоструктурный подход, т.к. он выявляет структурные уровни (ярусы) рельефа, дает их характеристику на основе так называемого «скелета» рельефа. Его внешний облик определяется закономерно расположенными точками и линиями, на которые как бы натянуты поверхности. Сочетания точек, линий и самих поверхностей создают в пространстве геометрические единства или комплексы. Выделяя их в каждом   ярусе рельефа, мы раскрываем их морфологическую структуру, а точки, линии и поверхности можно рассматривать в качестве ее элементов.
Число и характеристики элементов морфоструктуры не остаются постоянными, а меняются в зависимости от происхождения и возраста рельефа, связаны и с масштабом карты, и с процедурой генерализации рельефа. При увеличении масштаба карт некоторые точки и линии превращаются в поверхности и наоборот. Этим динамические элементы морфоструктуры отличаются от стандартных геометрических точек, линий и поверхностей. Для того чтобы сохранить данные различия, следует говорить о том, что морфометрия рельефа выделяет физические, а не геометрические точки, линии и поверхности, но изучает при этом их геометрические свойства и отношения в связи с анализом происхождения и развития рельефа и ландшафта.

…

%-----------------------------------
ЦИФРОВЫЕ МОДЕЛИ ПЛАСТИКИ РЕЛЬЕФА:
МЕТОДИКА ПОСТРОЕНИЯ И ВОЗМОЖНОСТИ ИСПОЛЬЗОВАНИЯ ПРИ ГЕОМОРФОЛОГИЧЕСКОМ АНАЛИЗЕ

С.Б. Кузьмин, Л.В. Данько, Е.А. Черкашин, Э.Ю. Осипов

Введение. Метод пластики рельефа

Анализ морфометрических показателей рельефа является важным звеном в цепи дистанционных методов, используемых при геоморфологических исследованиях, направленных на оценку современной ландшафтной ситуации. Исходными данными этого анализа являются цифровые модели рельефа, которые, как правило, строятся при помощи программных средств ГИС. В результате морфометрического анализа цифровая модель рельефа дополняется новыми слоями данных. В дальнейшем узловые участки созданных моделей рельефа проходят проверку с помощью полевых наблюдений, дешифрирования аэрофотографических и космических снимков [1-3].
При создании цифровых моделей рельефа и их интерпретации используются различные методы. В основу проводимого нами моделирования рельефа положены принципы метода пластики рельефа [4-6]. Он позволяет описывать топографическую поверхность с помощью изолиний высоты, горизонтальной, вертикальной и средней гауссовой кривизны, а также экстремумов кривизны – структурных линий рельефа (водораздел, тальвег, подошва, бровка). В результате генерализация рельефа, заложенная в горизонталях топографических карт, выделяется и становится приемлемой для морфометрического анализа инвариантной структуры рельефа. Метод основан на геометрическом преобразовании изогипс на основе «метода вторых производных». Вводится новая картографическая изолиния – плановой и профильной кривизны – морфоизографа, которая соединяет точки перегибов изогипс с одинаковой кривизной (нулевой). Она структурирует земную поверхность путем разделения на области дивергенции и конвергенции потоков вещества, создающие системную целостность – потоковые структуры. При этом графически обособляются водоразделы, склоны разной крутизны, экспозиции и протяженности. Метод пластики рельефа сравнительно-географический, поскольку объединяет точки (значения) топографической модели земной поверхности путем группировки по сходству признаков в математически однородные изолинии, изоповерхности, изообъемы.
При картографической реализации метода пластики рельефа области дивергенции и конвергенции отображают ориентированные гравитационным полем поверхности – повышения и понижения. При этом решается задача количественного картографического представления топографических факторов перераспределения вещества, его миграции и аккумуляции. Причем метод пластики даже на равнинных территориях позволяет выделить и изучить систему бассейнов стока, в пределах которых прослеживаются области формирования, транзита и аккумуляции твердого и водного стока. Метод позволяет также выявлять бассейны стока по следам древней гидрографической сети с областями формирования, транзита и аккумуляции вещества.
Методологической основой метода пластики рельефа является структурный подход [7,8], который направлен на установление структурных уровней и элементов рельефа, их характеристику на основе выявления и анализа структурных линий, «скелета» рельефа. Внешний облик рельефа определяется закономерными сочетаниями характерных точек, линий, на которые как бы натянуты поверхности. Сочетания точек, линий и поверхностей создают в пространстве геометрические единства (или комплексы). Выделяя их в каждой форме рельефа, мы раскрываем их морфологическую структуру, а точки, линии и поверхности можно рассматривать в качестве ее элементов [9-11].
Одной из особенностей элементов морфологической структуры является то, что их число и характеристики не остаются постоянными, а меняются в зависимости от происхождения и возраста рельефа, связаны и с масштабом карты, и с процедурой генерализации рельефа. При увеличении масштаба карт некоторые точки и линии превращаются в поверхности. При уменьшении масштаба карт наблюдаются обратные изменения. Этим элементы морфологической структуры отличаются от геометрических точек, линий и поверхностей. Для того чтобы сохранить эти различия, следует говорить о том, что морфометрия рельефа выделяет физические, а не геометрические точки, линии и поверхности, но изучает при этом их геометрические свойства и отношения в связи с анализом происхождения и развития рельефа.
Использование различных методов исследования рельефа: с помощью натурных описаний, по аэрокосмоснимкам, топографическим картам приводит к единому результату – установлению его инвариантов. Совокупность структурных линий и элементов рельефа образует его абстрактный инвариант. Применение абстрактного инварианта для анализа рельефа имеет универсальное значение, так как структура рельефа, запечатленная в его инвариантных линиях, во всех отношениях является определяющей [6].

Объекты и процедура исследований
Объектом исследований выбрана территория Приольхонья, которая расположена в Западном Прибайкалье (Рис.1). Она ограничивается береговой линией озера Байкал и водоразделом Приморского хребта между поселками Бугульдейка и Зама. Приольхонье почти полностью входит в состав Косостепско-Приольхонского округа, Прибайкальской провинции, Байкальско-Становой страны. Это определяет особенности климата и геолого-геоморфологического строения территории. Подробное геоморфологическое описание района исследований дано в многочисленных специальных работах [12-23]. Собственно моделируемый полигон расположен в районе залива Мухор между верховьями реки Кучелга и долиной реки Харга, и включает приводораздельные участки Приморского хребта, участок его макросклона, фрагменты Кучелго-Таловской депрессии и Приольхонского плато.
Процедура моделирования начинается с составления цифровой модели местности. Под цифровой моделью местности в нашей работе понимается определенная форма представления исходных данных и способ их структурного описания, позволяющий «вычислять» (восстанавливать) объект (рельеф) путем интерполяции, аппроксимации или экстраполяции [3]. Моделирование рельефа осуществлялось путем интерполяции оцифрованных изолиний с топографических карт, при этом, следует сказать, что варианты моделирования могут быть разгруппированы, исходя из принципа моделирования. Модели могут быть представленные в двух видах: TIN и GRID.
TIN – система неравносторонних треугольников, соответствующая триангуляции Делоне. Она используется в качестве модели данных при конструировании цифровой модели рельефа, представляя его набором высотных отметок в узлах сети, и заменяя его, тем самым, многогранной поверхностью. Кроме этого, модели TIN могут использоваться при генерации дополнительных данных при их нехватке для интерполяции. Преимущество триангуляционной модели заключается в отсутствии преобразований над исходными данными, но это не позволяет использовать ее для детального анализа.
GRID – представление цифровой модели в виде регулярной сетки квадратов, когда в ее узлах заданы значения показателя. Модели, полученные при интерполяции таким способом, представляют собой непрерывную матрицу данных, которая может быть подвергнута более тщательному анализу, поэтому данный способ моделирования и представления данных и был нами использован для создания цифровой модели рельефа.
Составление цифровых карт происходило в среде ГИС ArcInfo и ArcView и состояло из следующих основных этапов.
Сканирование на высокоточных планшетных сканерах.
Взаимоувязка  фрагментов. При использовании малоформатных сканеров карту сканируют фрагментами. Соседние фрагменты должны иметь перекрытия, необходимые для последующей «склейки». Из опыта работ можно сказать, что перекрытия соседних фрагментов карты должны составлять не менее 10-12 см. Эта процедура выполняется в программе EasyTrace.
Привязка растров. Для этого используются пересечения линий прямоугольной координатной сетки. Оптимальное количество контрольных точек для каждого фрагмента карты составляет 4-9 штук.
Векторизация. Осуществляется путем ручной или полуавтоматической (в зависимости от сложности карты) трассировки по растровой подложке с выделением отдельных тематических слоев. Каждый слой несет информацию об одном из аспектов исследуемой территории. В результате формируется точечная, либо линейная карта объектов. Векторизацию проводят в программах-векторизаторах: EasyTrace или R2V.
Экспорт векторных слоев в формат ГИС ArcView.
«Сшивка» карт. Две изолинии на разных картах проверяются на наличие одинаковых идентификаторов, а затем с помощью встроенных функций программы соединяются в одну изолинию.
ArcView – ГИС, обладающая необходимым набором средств для ввода, хранения и обработки пространственных данных. Функциональный набор операций ArcView расширяется при помощи множества дополнительных модулей, как внешних, так и реализованных в виде скриптов языка программирования Avenue. С помощью дополнительного модуля Spatial Analyst производится обработка данных, представленных в виде грид-поверхностей, и осуществляются запросы к этим данным. Кроме того, существует значительно обогащающие возможности ArcView модуль X-Tools. При помощи X-Tools выполняются типичные редакторские операции с темами.
Для анализа пластики рельефа, представленного в виде грида, использовалась программа OpenEV. Встроенный в программу инструментарий позволяет осуществлять программируемые в виде скриптов матричные преобразования над гридами. Программирование производилось в среде Python, состоящей из языка программирования и библиотеки функциональных модулей. Добавление модулей преобразования грид-данных – трудоемкая задача. OpenEV позволяет экспортировать получаемые грид-данные в большинство грид-форматов, в том числе в форматы, поддерживаемые ГИС ArcView.
В программе R2V оцифрован исследуемый участок местности на картах масштаба  1 : 25 000, отсканированных и сохраненных в формате *.tif. Выделены: изолинии высот, реки, береговая линия, высотные отметки и др.
Морфометрический анализ рельефа, представленного в виде грид-данных, отличается от анализа, использующего в качестве исходных данных горизонтали морфоизогипс. В основу анализа положено утверждение о том, что все точки грида разделяются на два класса: конвергенции и дивергенции. Считается, что точка принадлежит области (классу) конвергенции, если расположенные в окрестности этой точки линии стока (вектора) сходятся. Точка принадлежит области дивергенции, если расположенные в окрестности этой точки линии стока (вектора) расходятся [5].
При помощи библиотеки GDAL производится расчет поля векторов стока как приближение отношений разности уровней точек окрестности к расстоянию между этими точками (принята система координат, где ось y направлена вниз):








Далее анализируется схождение/расхождение векторов стока в окрестности точки, которое производится следующим образом. Векторы стока расходятся, если разность сумм проекций векторов стока снизу и справа, сверху и слева от центральной точки на ось x и y положительна. Если указанная сумма отрицательна, то векторы стока сходятся.
Процедура моделирования рельефа на основе метода пластики рельефа сводится к следующим шагам (подпрограммам).
Вычисление поля векторов стока. В результате формируются два массива: sx – x-составляющие поля векторов стока, sy – y-составляющие поля векторов стока. Составляющие поля векторов стока являются грид-данными.
Вычисление схождения/расхождения векторов стока. Вычисляется разность сумм проекций x-составляющих и y-составляющих векторов стока.
Анализ поля схождения/расхождения векторов стока. На поле схождения/расхождения, выделяются области, соответствующие знаку величины схождения/расхождения. Анализ осуществляется сравнением величины с 0.
Для обработки исходных GRID-данных, полученных при помощи интерполяции изолиний используется обобщение методики, описанной выше. Такие данные содержат искажения, оказывающие существенное влияние на качество конечного результата. Поле схождения/расхождения векторов стока вычисляется не по четырем векторам, а по дискретному контуру, являющемуся аппроксимацией окружности заданного радиуса с центром в искомой точке.
Окружность строится при помощи алгоритма Браземхэма. Вектор стока в точке окружности скалярно умножается на единичный вектор, направленный из этой точки в центр окружности, результаты умножения для всех точек окружности суммируются и усредняются. Дальнейшая процедура анализа пластики рельефа проводится аналогично.
Параметром к этой обобщённой методики является радиус окружности (в точках). Чем больше радиус, тем сильнее будет происходить усреднение значений, т.е. уменьшение влияния искажений GRID на конечный результат, однако разрешение искомого результата также уменьшается в количество раз равному радиусу окружности.
Геоморфологический анализ карт пластики рельефа проведем с помощью двух описанных выше моделей, которые для простоты изложения назовем условно «рельефX» и «рельефY» (Рис.2,3). Они отражают соответственно горизонтальную (плановую) и вертикальную (профильную) кривизну земной поверхности. Раскраска грид-данных отображает интенсивность стока по соответствующей координате в конкретной точке. Анализ будет проведен только для изображенной на рисунках части исследуемой территории Приольхонья. Экстраполяция данных – задача последующих исследований.

Анализ моделей по ярусам рельефа (структурный)
Водораздельный ярус представлен гривами и гребнями отрогов Приморского хребта, осложненными останцовыми формами. Их осевые линии дешифрируются с помощью модели «рельефX». Модель «рельефY» позволяет проследить главные направления сноса осадочного материала с гребней и грив, а также оценить площади и конфигурацию поверхностей сноса.
Склоновый ярус рельефа представлен приводораздельным пологим макросклоном Приморского хребта, крутым склоном-эскарпом вдоль Приморского разлома, пологим предгорным делювиальным шлейфом, а также отдельными небольшими склонами на участке Приольхонского плато.
Приводораздельный макросклон проявлен на обеих моделях рельефа достаточно отчетливо, причем заметно, что при приближении к склону-эскарпу, макросклон становится все более террасированным, с резкими (пусть и небольшими) перепадами высот. Отследить эти перепады с помощью топографических карт и даже при визуальном обследовании практически невозможно. Более того, модель «рельефY» позволяет на пологой верхней поверхности выделить зоны (полосы) преимущественной аккумуляции (заболоченные пространства) и сноса рыхлого материала, а на нижней террасированной поверхности даже подсчитать количество невысоких уступов (перепадов) и определить их морфометрические характеристики, что также возможно только с использованием данных моделей пластики рельефа. Модель «рельефX» позволяет проследить переходы уступов на водораздельных гривах и гребнях в перепады на макросклоне.
Склон-эскарп Приморского разлома также отчетливо проявлен на обеих моделях. Модель «рельефY» позволяет определить ориентацию и приблизительную ширину склона-эскарпа. Модель «рельефX» в этом смысле более информативна. Кроме указанных морфометрических характеристик, она позволяет определить количество отдельных уступов на склоне-эскарпе, их ширину, ориентацию и геометрические взаимоотношения (которые являются отражением позднекайнозойских тектонических движений). Эта же модель позволяет выделить делли (зарождающиеся распадки) на склоне-эскарпе, что в других случаях возможно только при визуальном обследовании.
Предгорный делювиальный склон (шлейф) Приморского разлома проявлен на моделях фрагментарно, поскольку прерывается по простиранию боковыми долинами-притоками реки Кучелга. Обе модели подчеркивают две разновозрастных генерации поверхностей делювиального склона-шлейфа. Более древняя генерация значительно положе и сохранилась только в виде небольших участков, слабо- или совсем нетеррасированных (может быть сопоставлена с эпохой среднеплейстоценового относительного тектонического покоя). Более молодая генерация круче и заметно террасирована (что особенно хорошо проявлено в модели «рельефY»). На ней отмечаются закладывающиеся совсем юные эрозионные врезы, которые можно сопоставить с эпохой тектонической активизации второй половины позднего плейстоцена.
Склоны Приольхонского плато также хорошо проявлены на обеих моделях и их анализ возможен для конкретных участков. Модель «рельефY» подходит для анализа экспозиционных различий склонов. Модель «рельефX» наиболее подходит собственно для морфометрического анализа: протяженность, ширина, ориентация и другие характеристики склонов (например, схождения, узловые сочленения, трансекции и др.).
Долинный ярус представлен террасами, высокой и низкой поймой реки Кучелга с левыми безымянными притоками и реки Харги.
Граница долинного яруса наиболее четко прослеживается на модели «рельефX». Выделяются крупные долины постоянных водотоков, трассирующие всю территорию, верховья которых расположены на Приморском хребте, и мелкие долины временных водотоков на Приольхонском плато. Модель «рельефY» позволяет выделить в долинном ярусе фрагменты террас, высокой и низкой (в основном заболоченной) поймы.
Аквальный ярус представлен подводными формами залива Мухор на модели «рельефY». Это дельтовые наносы в устьевой части реки Кучелга, волноприбойные косы на мысах, бар в волновой «тени» острова Тойнак.

Анализ моделей по происхождению рельефа (генетический)
По генезису преобладает эрозионно-денудационный, тектонический, гравитационный, карстово-суффозионный и эоловый рельеф. Широко представлены формы рельефа смешанного генезиса.
Эрозионно-денудационный рельеф дешифрируется на обеих моделях в виде долин постоянных и временных водотоков. Более информативна в этом смысле модель «рельефX». Эти геоморфологические элементы достаточно подробно рассмотрены в предыдущем разделе. Здесь отметим только, что среди денудационных форм выделяются фрагменты поверхностей выравнивания, древних структурно-террасовых уровней и собственно речных долин (дочетвертичных или среднеплейстоценовых), отрезанных от современного базиса эрозии и эрозионной сети, и разрушающихся ныне в основном за счет гравитационных процессов (подробнее будут рассмотрены ниже).
Тектонический рельеф представлен широким многообразием форм за счет того, что исследуемый участок расположен в зоне активного Приморского разлома. Морфоструктурный анализ проведен в основном по модели «рельефX». Прежде всего, отмечается собственно осевая зона Приморского разлома. Она сложно построенная, состоит из серии более мелких тектонических уступов, западин и шерлопов. Характер рельефа свидетельствует о том, что Приморский разлом является сбросом. Вместе с тем, отмечается наличие сдвиговой компоненты в кинематике тектонических движений по разлому.
Ранее нами отмечалась только лево-сдвиговая компонента [19]. В работах С.И. Шермана [24,25] высказывалась мысль о право-сдвиговых движениях по разлому. Анализ модели «рельефX» приводит нас к мысли о переменном знаке сдвиговых дислокаций по разлому, по крайней мере в плиоцен-четвертичное время. На модели «рельефХ» наиболее крупные долины (плиоцен?) имеют левый сдвиг, менее крупные долины (ранний-средний плейстоцен?) имеют правый сдвиг, и, наконец, самые мелкие и молодые элементы рельефа (поздний плейстоцен?) вновь приобретают левый сдвиг.
Хорошо проявлены и другие древние геолого-структурные швы, например, зона Чернорудско-Баракчинского разлома. В восточной части исследуемого участка она подчеркнута небольшими линейно вытянутыми эскарпами. Особенно примечателен Мухорский эскарп, являющийся почти точной копией эскарпа Приморского разлома в миниатюре. Анализ структурного рисунка зоны Мухорского и параллельных ему эскарпов позволяет также говорить о наличии лево-сдвиговой компоненты в самых молодых рифтогенных разломах Приольхонья. На модели «рельефХ» впервые однозначно зафиксированы правосторонние сдвиги по разломам, поперечным основным рифтовым структурам (например, на севере участка вдоль молодой гривы).
Модель «рельефХ» позволяет оконтурить тектонические блоки, разделенные активными разломами. Иногда можно определить кинематику и характер тектонических движений блока. Например, на западе участка правым сдвиго-сбросом вычленен узкий полосообразный блок, испытывающий активное опускание, совмещенное с кручением по часовой стрелке.
Более детальный анализ на отдельных фрагментах позволяет выделить и другие, не менее интересные тектонически обусловленные формы рельефа.
Гравитационный рельеф во многом повторяет тектонический, хотя встречаются и своеобразные геоморфологические элементы. Прежде всего, это шлейфы рыхлого материала у подножья эскарпов. Гравитационное разрушение эскарпов протекает настолько активно, что мощность шлейфов даже под невысокими эскарпами достигает больших величин. Например, у подножья Мухорского эскарпа высотой всего 20-30 м мощность шлейфа превышает 1.8 м (заложенный шурф на этой глубине не встретил кристаллических пород, что позволяет предполагать и большую величину мощности рыхлых отложений). Оползневые гравитационные структуры дешифрируются на обеих моделях на предгорном делювиальном шлейфе Приморского хребта. Обвально-осыпные формы отмечаются на модели «рельефХ» в продольных профилях распадков, поперечных р.Кучелга. На модели «рельефY» фиксируется конус выноса крупно-глыбового материала в долине р.Харга.
Карстово-суффозионный рельеф отмечается на участке Приольхонского плато на модели «рельефХ». В основном это провальные котловины и небольшие суффозионные воронки. В первом случае следует отметить также карстовые пещеры. В последнем случае невозможно обойти вниманием карстовые провалы, совмещенные со складчатыми структурами.
Эоловый рельеф проявлен дефляционными формами в виде котловин выдувания и останцов обтачивания на участке Приольхонского плато. Ярко выраженный останец обтачивания расположен в центральной части, а две котловины выдувания – в восточной части исследуемой территории.
Рельеф смешанного генезиса выделяется на обеих моделях. Формы его разнообразные, поэтому приведем лишь несколько примеров. Упомянутые выше котловины выдувания сформированы над карстовыми внутригорными полостями, в которые происходили гравитационные провалы, а затем эти элементы оформлялись дефляцией. Рядом с котловинами, несколько юго-западнее расположена карстово-тектоническая замкнутая депрессия в виде грабена. Здесь внутригорная карстово-суффозионная полость способствовала тектоническому обрушению прямоугольного блока. Причем это обрушение произошло сравнительно недавно, поскольку на дне депрессии еще не успел сформироваться слой рыхлого материала. Флювиально-гравитационно-тектонические оползни дешифрируются на предгорном шлейфе Приморского хребта, где их подновлению периодически способствуют землетрясения. Прибрежные косы и бары залива Мухор формируются из песчаного материала, поставляемого, главным образом, эоловыми процессами.

Анализ реликтовых форм рельефа
Модель «рельефХ» и отчасти модель «рельефY» позволяют анализировать реликтовые формы рельефа, такие как поверхности выравнивания, древние террасовые уровни, участки брошенных древних речных долин.
Фрагмент раннеплиоценовой (?) поверхности выравнивания прослеживается на обеих моделях в приводораздельной части Приморского хребта. На севере имеется еще один менее выразительный фрагмент этой поверхности. Хотя, по всей видимости, вся приводораздельная часть Приморского хребта (с гривами, пологими склонами и выровненными поверхностями) может рассматриваться как реликт раннеплиоценового рельефа. Этого же возраста поверхности выравнивания прослеживаются и на участке Приольхонского плато, где они занимают небольшие площади и расположены мозаично.
Реликтовые террасовые уровни (два?) прослеживаются в восточной части исследуемой территории, где они привязаны, вообще говоря, к брошенной древней речной долине. Террасы не только структурные, но и аккумулятивные, о чем свидетельствует мощная толща (более 1 м) рыхлых отложений песчаного и песчано-гравелистого материала на их поверхности. Возраст террас и долины можно сопоставить с эпохой тектонического покоя среднего плейстоцена. Сток по этой древней брошенной долине осуществлялся с северо-востока на юго-запад. В современной речной сети такое направление стока для территории Приольхонья не встречается. Это позволяет думать, что фаза тектонической активизации позднего плейстоцена привела к существенным геоморфологическим преобразованиям на исследуемом участке.
Северо-западнее, на предгорном шлейфе Приморского хребта, в седловине между двумя распадками также отмечаются реликты среднеплейстоценового (?) рельефа. С юго-запада это фрагмент поверхности выравнивания. С северо-востока, на склоне седловины, прилегающем к долине р. Харга, отмечаются реликтовые террасовые уровни (вероятно только структурные, поскольку обнаружить на них аллювий не удалось). Можно проследить даже излучину (меандр) древней реки. Террасовых уровней здесь, по крайней мере, три, и там, где заканчивается самый нижний из них, расположен резкий крутой (обрывистый) склон к современной долине р. Харга.
Реликты древних долин имеются и в южной части исследуемой территории, где они также подтверждают существование еще в раннем плейстоцене речного стока с северо-востока на юго-запад. Поскольку они небольшие, то нет основания распространять данный вывод на все Приольхонье, пока не будут получены дополнительные материалы на других участках.
Заключение
Таким образом, геоинформационное моделирование на основе метода пластики рельефа показало его высокую информативность для проведения геоморфологических исследований. Следует отметить главную особенность моделей «рельефX» и «рельефY», которая заключается в том, что на первой модели более четко проявлены индикационные признаки эндогенного (тектонического) рельефа, а на второй – экзогенного рельефа (хотя элементы всех типов рельефа имеются и на обеих моделях). Модели могут использоваться как совместно, так и по отдельности (в зависимости от конкретной задачи), но в любом случае они должны рассматриваться как взаимодополняющие друг друга при дистанционных геоморфологических построениях.

Список литературы
1. Коновалова Н.А., Капралов Е.Г. Введение в ГИС. Петрозаводск: Изд-во ПГУ, 1995. 148 с.
2. Жуков В.Т., Новаковский Б.А., Чумаченко А.М. Компьютерное геоэкологическое картографирование. М.: Научный мир, 1999. 128 с.
3. Новаковский Б.А., Прасолов С.В., Прасолова А.И. Цифровые модели рельефа реальных и абстрактных геополей. М.: Научный мир, 2003. 64 с.
4. Степанов И.Н., Абдуназаров У.К., Брынских М.Н. и др. Временная методика по составлению карт пластики рельефа крупного и среднего масштаба. Методические рекомендации. Пущино: ОНТИ НЦБИ АН СССР, 1983. 112 с.
5. Метод пластики рельефа в тематическом картографировании. Пущино: ОНТИ НЦБИ АН СССР, 1987. 160 с.
6. Геометрия структур земной поверхности. Пущино, 1991. 202 с.
7. Ласточкин А.Н. Рельеф земной поверхности. Ленинград: Недра, 1991. 240 с.
8. Симонов Ю.Г. Объяснительная морфометрия рельефа. М.: ГЕОС, 1999. 263 с.
9. Пириев Р.Х. Методы морфометрического анализа рельефа. Баку: Элм, 1986. 117с.
10. Поздняков А.В., Черванев И.Г. Самоорганизация в развитии форм рельефа. М.: Наука, 1990. 202 с.
11. Якименко Э.Л. Морфометрия рельефа и геология. Новосибирск: Наука, 1990. 201 с.
12. Логачев Н.А, Ломоносова Т.К., Климанова В.М. Кайнозойские отложения Иркутского амфитеатра. М.: Наука, 1964. 195 с.
13. Флоренсов Н.А. Байкальская рифтовая зона и некоторые задачи ее изучения // Байкальский рифт. М.: Наука, 1968. С.40-56.
14. Нагорья Прибайкалья и Забайкалья. М: Наука, 1974. 360 c.
15. Плешанов С.П., Ромазина А.А. Основные этапы формирования рельефа Приольхонья // Геоморфология. 1975. № 4. С.85-89.
16. Тайсаев Т.Т. Эоловые процессы в Приольхонье и на о.Ольхон (Западное Прибайкалье) //  Доклады АН СССР. 1982. № 4. С.948-951.
17. Уфимцев Г.Ф. О неотектонике Приольхонья // Геология и геофизика. 1985. № 6. С.37-45.
18. Васянович А.В. Геоморфологические особенности территории Прибайкальского национального парка // География и природные ресурсы. 1990. № 4. С.67-77.
19. Кузьмин С.Б. Геоморфология зоны Приморского разлома (Западное Прибайкалье) // Геоморфология. 1995. № 4. С.53-61.
20. Кузьмин С.Б. Инженерно-экологическая оценка рельефа Приольхонья в рекреационно-туристических целях // Инженерная экология. 2002. № 6. С.41-53.
21. Макаров С.А. Геоморфологические процессы Приольхонья в голоцене // География и природные ресурсы. 1997. № 1. С.77-84.
22. Аржанникова А.В., Гофман Л.Е. Проявления неотектоники в зоне влияния Приморского разлома // Геология и геофизика. 2000. № 6. С.811-818.
23. Снытко В.А., Данько Л.В., Кузьмин С.Б. и др. Разнообразие геосистем контакта тайги и степи западного побережья Байкала // География и природные ресурсы. 2001. № 2. С.61-68.
24. Шерман С.И. Приморский разлом в Западном Прибайкалье // Информационный бюллетень ИЗК СО АН СССР (1967-1968 г.г.). Иркутск, 1970. С.14-15.
25. Шерман С.И. Физические закономерности развития разломов земной коры. Новосибирск: Наука, 1977. 101 с.






Подписи к рисункам

Рис. 1. Район исследований – Приольхонье (показан штриховкой).

Рис. 2. Цифровая модель пластики рельефа Приольхонья – рельеф X.

Рис. 3. Цифровая модель пластики рельефа Приольхонья – рельеф У.

%-----------------------------------

%\bibliographystyle{splncs03}
%\bibliography{example}
 \begin{thebibliography}{99}

 \end{thebibliography}

%%%%%%%%%%%%%%%%%%%%%%%%%%%%%%%%%%%%%%%%%
%% For the final version of the paper: %%
%%%%%%%%%%%%%%%%%%%%%%%%%%%%%%%%%%%%%%%%%

%% Author information
%\vspace{4ex}\noindent
%\textbf{Author One} is\dots
%
%\bigskip\noindent
%\textbf{Author Two} is\dots
%
%\bigskip\noindent
%\textbf{Author Three} is\dots

%% Reception and acceptance information
%\bigskip
%\paragraph{Received: Month DD, 20YY; Accepted: Month DD, 20YY.}

\end{document}

%%% Local Variables:
%%% mode: latex
%%% TeX-master: t
%%% End:
